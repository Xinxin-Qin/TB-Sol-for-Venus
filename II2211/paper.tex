\documentclass[10pt,twocolumn]{article}

% Pacchetti per font e formattazione
\usepackage{times}        % Times New Roman
\usepackage[utf8]{inputenc}
\usepackage[T1]{fontenc}
\usepackage{geometry}
\geometry{a4paper, margin=2cm}
\usepackage{titlesec}
\usepackage{multicol}
\usepackage{cite}

% Titoli sezioni (formato)
\titleformat{\section}{\bfseries\fontsize{12}{14}\selectfont}{\thesection.}{0.5em}{}
\titleformat{\subsection}{\bfseries\fontsize{11}{13}\selectfont}{\thesubsection.}{0.5em}{}

% Dati frontespizio
\title{\textbf{TB for Venus RISC-V}\\[0.5em]}

\author{Chinmay Purandare\\
Lorenzo Parata\\
Department of Electronics\\
KTH Royal Institute of Technology\\
Stockholm, Sweden\\
Emails: \{chinmayp\}@kth.se, \{parata\}@ug.kth.se}

\date{}

\begin{document}

\maketitle

\section*{Abstract}
<Contains short summary of the paper>\\
Main body – Double column format\\
Text – Font – Times New Rom, 10 pts\\
Section Headers – Bold Font – Times New Roman, 12 pts\\
Subsection Headers – Bold font – Times new Roman, 11 pts\\
Paper length target 4-6 pages.

\section*{Keywords}
<optional section to help search engines to find the paper>

% Sezioni principali
\section{Introduction}
<Answers the question: Why do I do this research? The section may also contain which methodology have been used? Any research ethics? Sustainability issues?>

\section{Related Work}
This section reviews research relevant to developing a testbench for bit-serial RISC-V processors targeting Venus surface applications. We examine RISC-V architectures, high-temperature electronics, FPGA implementation methods, and verification frameworks. These foundations inform the design of a testbench capable of validating processor functionality under extreme environmental constraints.

\subsection{RISC-V}
Researchers from TU graz developed FazyRV \cite{kissich2024fazyrv}, a minimal-area open-source RV32I RISC-V core targeting IoT and low-workload applications, addressing the problem that 32-bit RISC-V processor cores reach a boundary on their minimal size. Their goal was to minimize area demand while fulfilling performance requirements and close the gap between prevalent 32-bit and 1-bit-serial RISC-V cores.\\
Qian Wei and his colleagues created a comprehensive survey of RISC-V instruction set architecture extensions because while RISC-V is popular for embedded processors \cite{cui2023riscv}, there is still a gap between RISC-V's capabilities and the requirements of various emerging computing scenarios like artificial intelligence and cloud computing.\\
Gautschi et al. conducted a comprehensive comparison of ultra-low-power RISC-V cores for Internet-of-Things applications \cite{gautschi2017slow}, analyzing the trade-offs between performance and energy efficiency in resource-constrained environments. Their work provides valuable insights into processor design considerations for applications with tight area and power constraints, which directly supports the rationale for bit-serial processor implementations in challenging deployment scenarios such as the 12-pad limitation required for Venus surface operations.

\subsection{Processors for Venus}
Current research on high-temperature electronics for Venus has shown that SiC ICs can sustain operation for over a year at 500 °C and for months in simulated Venus conditions, supporting sensors and simple microprocessors. This demonstrates the feasibility of long-duration surface missions and motivates concepts like LLISSE targeting low-power seismic monitoring \cite{kim2022sic}. \\
At the processor level, studies of SiC-based computing infrastructures reveal that current prototypes achieve significantly lower throughput than space-proven silicon systems, yet provide guidelines at the microarchitecture and ISA levels for developing processors able to operate under Venus's extreme environment \cite{kremic2021temperature}.\\
Pradhan and colleagues provided a comprehensive review of materials for high-temperature digital electronics \cite{pradhan2024materials}, highlighting the challenges and solutions for electronic systems operating at temperatures as high as 500°C and beyond. Their work emphasizes the critical importance of developing new material solutions beyond conventional silicon-based devices for applications including space exploration, with specific attention to the extreme conditions encountered in Venus surface missions.

\subsection{FPGA}
One verification strategy is co-emulation, where the RTL design on the FPGA is run in parallel with a trusted software model, such as the Spike simulator \cite{moreno2023fpga}, on a host PC. This allows for high-speed verification by comparing the core's architectural state against the simulator in real-time. Furthermore, using FPGAs with RISC-V is advantageous as it allows for optimized hardware, where the FPGA is configured with only the peripherals required for a specific application \cite{efinix2023riscv}.

\subsection{FPGA-Based Verification Frameworks}

Kim \cite{kim2019fpga} demonstrated the importance of FPGA-based acceleration for RTL evaluation by introducing automated flows that generate cycle-accurate simulators directly from RTL. This approach reduces the engineering effort compared to earlier manual FPGA frameworks and enables both faster and more reliable pre-silicon verification. 

Building on this direction, Qin \textit{et al.} \cite{qin2025feriver} proposed FERIVer, an FPGA-assisted framework for verifying RISC-V processors. Their method exploits the heterogeneous architecture of SoC FPGAs, running an instruction set simulator on the processing system in parallel with the RTL core on the programmable logic. By cross-checking execution states, FERIVer achieves significant speedups over traditional software simulators while maintaining accuracy. 

Together, these works highlight how FPGA-based infrastructures provide an effective foundation for validating processor designs before costly SoC fabrication, a goal directly aligned with the testbench methodology developed in this project.


\section{Specification}
<Outlines the constraints this work has, if any>\\

\section{Architecture}
<How do I/we plan to build this gadget given the constraints I have?. This section is sometimes split into subsections like Hardware and Software>

\section{Experiments}
<What experiments have I done? This is more or less a specification of the experiments.>

\section{Results}
<What results did I get from the experiments. This section is sometimes merged with the experiments section>

\section{Conclusion}
<This is a retrospective look at the introduction, and should present answers to the questions asked there and throughout the paper.>

\subsection{Future Work}
<Optional section that describe possible future work>

\bibliographystyle{ieeetr}
\bibliography{references}

\end{document}