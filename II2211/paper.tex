\documentclass[10pt,twocolumn]{article}

% Pacchetti per font e formattazione
\usepackage{times}        % Times New Roman
\usepackage[utf8]{inputenc}
\usepackage[T1]{fontenc}
\usepackage{geometry}
\geometry{a4paper, margin=2cm}
\usepackage{titlesec}
\usepackage{multicol}
\usepackage{cite}

% Titoli sezioni (formato)
\titleformat{\section}{\bfseries\fontsize{12}{14}\selectfont}{\thesection.}{0.5em}{}
\titleformat{\subsection}{\bfseries\fontsize{11}{13}\selectfont}{\thesubsection.}{0.5em}{}

% Dati frontespizio
\title{\textbf{Project Title}\\[0.5em]
\large Project subtitle if any}

\author{
Author #1 (auth #1 Email address), Author #2 (Email address), …
}

\date{}

\begin{document}

\maketitle

\section*{Abstract}
<Contains short summary of the paper>\\
Main body – Double column format\\
Text – Font – Times New Rom, 10 pts\\
Section Headers – Bold Font – Times New Roman, 12 pts\\
Subsection Headers – Bold font – Times new Roman, 11 pts\\
Paper length target 4-6 pages.

\section*{Keywords}
<optional section to help search engines to find the paper>

% Sezioni principali
\section{Introduction}
<Answers the question: Why do I do this research? The section may also contain which methodology have been used? Any research ethics? Sustainability issues?>

\section{Related Work}
<Answers the question: What have others done previously in do this research topc?>\\
<Literature study, refer to each reference [1] where it makes sense>
\subsection{riscv}
Researchers from TU graz developed FazyRV[1], a minimal-area open-source RV32I RISC-V core targeting IoT and low-workload applications, addressing the problem that 32-bit RISC-V processor cores reach a boundary on their minimal size. Their goal was to minimize area demand while fulfilling performance requirements and close the gap between prevalent 32-bit and 1-bit-serial RISC-V cores.\\
Qian Wei and his colleagues created a comprehensive survey of RISC-V instruction set architecture extensions because while RISC-V is popular for embedded processors [2], there is still a gap between RISC-V's capabilities and the requirements of various emerging computing scenarios like artificial intelligence and cloud computing.

\subsection{processors for venus}
Current research on high-temperature electronics for Venus has shown that SiC ICs can sustain operation for over a year at 500 °C and for months in simulated Venus conditions, supporting sensors and simple microprocessors. This demonstrates the feasibility of long-duration surface missions and motivates concepts like LLISSE targeting low-power seismic monitoring.[3] \\
At the processor level, studies of SiC-based computing infrastructures reveal that current prototypes achieve significantly lower throughput than space-proven silicon systems, yet provide guidelines at the microarchitecture and ISA levels for developing processors able to operate under Venus’s extreme environment.[4]\\
\subsection{FPGA}
One verification strategy is co-emulation, where the RTL design on the FPGA is run in parallel with a trusted software model, such as the Spike simulator[5], on a host PC. This allows for high-speed verification by comparing the core's architectural state against the simulator in real-time. Furthermore, using FPGAs with RISC-V is advantageous as it allows for optimized hardware, where the FPGA is configured with only the peripherals required for a specific application[6].

\subsection{FPGA-Based Verification Frameworks}

Kim [7] demonstrated the importance of FPGA-based acceleration for RTL evaluation by introducing automated flows that generate cycle-accurate simulators directly from RTL. This approach reduces the engineering effort compared to earlier manual FPGA frameworks and enables both faster and more reliable pre-silicon verification. 

Building on this direction, Qin \textit{et al.} [8] proposed FERIVer, an FPGA-assisted framework for verifying RISC-V processors. Their method exploits the heterogeneous architecture of SoC FPGAs, running an instruction set simulator on the processing system in parallel with the RTL core on the programmable logic. By cross-checking execution states, FERIVer achieves significant speedups over traditional software simulators while maintaining accuracy. 

Together, these works highlight how FPGA-based infrastructures provide an effective foundation for validating processor designs before costly SoC fabrication, a goal directly aligned with the testbench methodology developed in this project.


\section{Specification}
<Outlines the constraints this work has, if any>\\

\section{Architecture}
<How do I/we plan to build this gadget given the constraints I have?. This section is sometimes split into subsections like Hardware and Software>

\section{Experiments}
<What experiments have I done? This is more or less a specification of the experiments.>

\section{Results}
<What results did I get from the experiments. This section is sometimes merged with the experiments section>

\section{Conclusion}
<This is a retrospective look at the introduction, and should present answers to the questions asked there and throughout the paper.>

\subsection{Future Work}
<Optional section that describe possible future work>

\section*{References}

[1] M. Kissich and M. C. Baunach, “FazyRV: Closing the Gap between 32-Bit and Bit-Serial RISC-V Cores with a Scalable Implementation,” in *Proc. 21st ACM Int. Conf. on Computing Frontiers (CF ’24)*, Ischia, Italy, May 2024, pp. 240–248, doi: 10.1145/3625135.3655876.  

[2] E. Cui, T. Li, and Q. Wei, “RISC-V Instruction Set Architecture Extensions: A Survey,” *IEEE Access*, vol. 11, pp. 24696–24711, 2023, doi: 10.1109/ACCESS.2023.3246491.  

[3] H. Kim, J. Bagherzadeh, and R. G. Dreslinski, “SiC Processors for Extreme High-Temperature Venus Surface Exploration,” in *Proc. Design, Automation \& Test in Europe Conf. \& Exhibition (DATE)*, Antwerp, Belgium, 2022, pp. 406–411, doi: 10.23919/DATE54114.2022.9774769.  

[4] T. Kremic, “High Temperature Electronics for Venus Surface Applications: A Summary of Recent Technical Advances,” *The Planetary Science Journal*, vol. 2, no. 1, 2021, doi: 10.3847/25C2CFEB.E3883E19.  

[5] A. Moreno *et al.*, “An FPGA Verification, Debug and Performance Platform for RISC-V Cores,” in *Proc. RISC-V Summit Europe*, Barcelona, Spain, 2023.  

[6] Efinix Inc., “RISC-V and FPGAs,” *Efinix Blog*, Jul. 2023. [Online]. Available: https://www.efinixinc.com/blog/riscv-and-fpgas.html. [Accessed: Sep. 26, 2025].  

[7] D. Kim, *FPGA-Accelerated Evaluation and Verification of RTL Designs*, Ph.D. dissertation, Dept. Comput. Sci., Univ. California, Berkeley, Berkeley, CA, USA, Tech. Rep. UCB/EECS-2019-57, May 2019.  

[8] K. Qin, X. Guo, M. Schulz, and C. Trinitis, “FERIVer: An FPGA-assisted Emulated Framework for RTL Verification of RISC-V Processors,” *arXiv preprint* arXiv:2504.05284, Apr. 2025.  

\end{document}

